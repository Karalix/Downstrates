\documentclass[11pt]{beamer}
\usepackage{fontspec}
\usetheme{Copenhagen}
\begin{document}
	\author[M2 HCI 2016-2017]{Alix DUCROS \& Damien Gabriel}
	\title{Downstrates: A collaborative markdown editor}
	
	%\logo{}
	%\institute{}
	\date{}
	%\subject{}
	%\setbeamercovered{transparent}
	%\setbeamertemplate{navigation symbols}{}
	
	\begin{frame}[plain]
		\maketitle
		
%		\begin{exampleblock}{Summary}
%			Create a simple collaborative markdown editor with data persistence
%		\end{exampleblock}
%		\vspace{10pt}
%		
%		{\scriptsize  \texttt{Klokmose, C. N., Eagan, J. R., Baader, S., Mackay, W., \& Beaudouin-Lafon, M. (2015, November). Webstrates: shareable dynamic media. In Proceedings of the 28th Annual ACM Symposium on User Interface Software \& Technology (pp. 280-290). ACM.}}
	\end{frame}
	
	\begin{frame}{Introduction}
		\begin{itemize}
			\item[What ?] Collaborative markdown editor
			\item[Why ?] Learn CSCW tools hand-on
		\end{itemize}
	\end{frame}
	
	\begin{frame}{Features}
		\begin{itemize}
			\item Markdown editor
			\item Real-time
			\item Shareable button
			\item Data persistence
		\end{itemize}
	\end{frame}

\begin{frame}{Demonstration}
	place holder for image
	\begin{center}
		{\Huge And now, the demonstration!}
	\end{center}
\end{frame}

\begin{frame}{usage summary}
	As written in the README.md file:
	\begin{enumerate}
		\item launch a mongo instance
		(on system using systemctl: systemctl start mongodb)
		\item download all server and client required modules
		(npm install)
		\item build the client bundle
		(npm run build)
		\item launch the server
		(npm run start)
		\item Open a webrowser to http://localhost:8080
	\end{enumerate}
\end{frame}

\begin{frame}{Tools used}
	\begin{description}
		\item[Language:] Javascript (including client side using nodejs)
		\item[Server:] express \& socket.io
		\item[Markdown:]
		\item[Real-time:] sharedb
		\item[Data persistence:] sharedb-mongo
	\end{description}
\end{frame}

\begin{frame}{What is working}
	Everything ;)
\end{frame}

\begin{frame}{What can be improved?}
	Nothing ;)
\end{frame}
\end{document}
